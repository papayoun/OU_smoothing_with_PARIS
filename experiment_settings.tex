\PassOptionsToPackage{unicode=true}{hyperref} % options for packages loaded elsewhere
\PassOptionsToPackage{hyphens}{url}
%
\documentclass[]{article}
\usepackage{lmodern}
\usepackage{amssymb,amsmath}
\usepackage{ifxetex,ifluatex}
\usepackage{fixltx2e} % provides \textsubscript
\ifnum 0\ifxetex 1\fi\ifluatex 1\fi=0 % if pdftex
  \usepackage[T1]{fontenc}
  \usepackage[utf8]{inputenc}
  \usepackage{textcomp} % provides euro and other symbols
\else % if luatex or xelatex
  \usepackage{unicode-math}
  \defaultfontfeatures{Ligatures=TeX,Scale=MatchLowercase}
\fi
% use upquote if available, for straight quotes in verbatim environments
\IfFileExists{upquote.sty}{\usepackage{upquote}}{}
% use microtype if available
\IfFileExists{microtype.sty}{%
\usepackage[]{microtype}
\UseMicrotypeSet[protrusion]{basicmath} % disable protrusion for tt fonts
}{}
\IfFileExists{parskip.sty}{%
\usepackage{parskip}
}{% else
\setlength{\parindent}{0pt}
\setlength{\parskip}{6pt plus 2pt minus 1pt}
}
\usepackage{hyperref}
\hypersetup{
            pdftitle={Smoothing in partially observed OU process},
            pdfauthor={Pierre Gloaguen},
            pdfborder={0 0 0},
            breaklinks=true}
\urlstyle{same}  % don't use monospace font for urls
\usepackage[margin=1in]{geometry}
\usepackage{graphicx,grffile}
\makeatletter
\def\maxwidth{\ifdim\Gin@nat@width>\linewidth\linewidth\else\Gin@nat@width\fi}
\def\maxheight{\ifdim\Gin@nat@height>\textheight\textheight\else\Gin@nat@height\fi}
\makeatother
% Scale images if necessary, so that they will not overflow the page
% margins by default, and it is still possible to overwrite the defaults
% using explicit options in \includegraphics[width, height, ...]{}
\setkeys{Gin}{width=\maxwidth,height=\maxheight,keepaspectratio}
\setlength{\emergencystretch}{3em}  % prevent overfull lines
\providecommand{\tightlist}{%
  \setlength{\itemsep}{0pt}\setlength{\parskip}{0pt}}
\setcounter{secnumdepth}{0}
% Redefines (sub)paragraphs to behave more like sections
\ifx\paragraph\undefined\else
\let\oldparagraph\paragraph
\renewcommand{\paragraph}[1]{\oldparagraph{#1}\mbox{}}
\fi
\ifx\subparagraph\undefined\else
\let\oldsubparagraph\subparagraph
\renewcommand{\subparagraph}[1]{\oldsubparagraph{#1}\mbox{}}
\fi

% set default figure placement to htbp
\makeatletter
\def\fps@figure{htbp}
\makeatother


\title{Smoothing in partially observed OU process}
\author{Pierre Gloaguen}
\date{25/11/2020}

\begin{document}
\maketitle

\hypertarget{true-model}{%
\section{True model}\label{true-model}}

\begin{align}
\text{d} X(t) &= -\rho(X(t) - \mu) \text{d}t + \sigma \text{d}W(t),~X(0) = x(0)~t\geq 0\\
Y(k) &= X(k) + \sigma_{obs}\varepsilon_k,~ \varepsilon_k \overset{i.i.d}{\sim} \mathcal{N}(0, 1)
\end{align}

\begin{itemize}
\tightlist
\item
  Data are generated according to this true model, giving a vector
  \(y_{0:T}\) from which inference must be performed.
\end{itemize}

\hypertarget{biased-model}{%
\section{Biased model}\label{biased-model}}

We consider a bias version of the true model. This model is biased in
two ways:

\begin{enumerate}
\def\labelenumi{\arabic{enumi}.}
\tightlist
\item
  In the dynamics model, which is approximated through an Euler scheme
  (and setting that \(t_{k + 1} = t_k + \delta\)).
\item
  In the observation model, where it is supposed that only a fraction
  \(\alpha_{obs}\) of \(X(t_k)\) is observed. \begin{align}
  X(t_k + \delta) &= -\rho(X(t_k) - \mu)\times \delta + \sigma\sqrt{\delta} \times \tilde{\varepsilon}_k, X(0) = x(0), ~k =  0, 1,..., \tilde{\varepsilon}_k \overset{i.i.d}{\sim} \mathcal{N}(0, 1)\\
  Y(t_k) &= \alpha_{obs}X(t_k) + \sigma_{obs}\varepsilon_k,~k = 0, 1, ..., \varepsilon_k \overset{i.i.d}{\sim} \mathcal{N}(0, 1)
  \end{align}
\end{enumerate}

We here emphasize that this scenario is actually quite common, for
instance in ecology, where the continuous time dynamics model is
approximated in discrete time, and supposed observation model is subject
to some bias (we refer the interested reader to the literature of state
space models for population dynamics).

\hypertarget{smoothing}{%
\section{Smoothing}\label{smoothing}}

\begin{itemize}
\item
  In this context, our goal is to estimate, for each \(n\leq T\) the
  realization on \(y_{0:n}\) of the random variable:
  \[\Phi_n = \mathbb{E}\left[\sum_{k = 0}^{n} X(t_k)\vert Y_{0:n}\right].\]
\item
  It is known that in this context, the actual marginal distribution is
  Gaussian with parameters that can be obtained through Kalman
  recursions. The quantity \(\Phi_n\) can then be computed explicitly.
\item
  We consider three alternatives estimation of \(\Phi_n\) which are:

  \begin{itemize}
  \tightlist
  \item
    \emph{PaRIS:} The online estimation is obtained with the PaRIS
    method. The model used in the recursions is the true one, and the
    proposal is the optimal filter;
  \item
    \emph{Biased PaRIS:} The online estimation is obtained with the
    PaRIS method. However, here, the model used in the recursions is the
    biased one, and the proposal is the optimal filter relative to this
    biased model;
  \item
    \emph{Biased Kalman:} The estimation obtained with the Kalman
    recursions when the true model is replaced by the biased model;
  \end{itemize}
\item
  For each estimator \(\hat{\Phi}_n\), we compute
\item
  The scenario is performed with
  \(T = 100,~\delta = 0.5,~\rho = 1,~\mu = -5,~x(0) = 10,~\sigma = 0.5,~\sigma_{obs} = 0.7,~\alpha_{obs} = 0.95\),
  using 200 particles for the particle filter, and obtaining two
  backward samples at each backward sampling step.
\end{itemize}

\hypertarget{results}{%
\section{Results}\label{results}}

\begin{center}\includegraphics{experiment_settings_files/figure-latex/plot_results-1} \end{center}

We can see here that using the biased model (either with Kalman
recursions or PaRIS), the bias is our estimate only grows linearly with
\(n\), as expected by results EQREF. In this context, the bias due to
Monte Carlo error is not visible.

\hypertarget{the-mathbfl_n-kernel-in-hierarchical-linear-gaussian-model}{%
\section{\texorpdfstring{The \(\mathbf{L}_n\) kernel in hierarchical
linear Gaussian
model}{The \textbackslash{}mathbf\{L\}\_n kernel in hierarchical linear Gaussian model}}\label{the-mathbfl_n-kernel-in-hierarchical-linear-gaussian-model}}

Suppose we have observations \(y_1,\dots, y_n\) recorded at ordered
times \(\mathbf{T} = (t_1,\dots, t_n)\) which are supposed to be
realizations of the following hierarchical linear Gaussian model:

\begin{align}
X(t + \delta) &= a_\delta X(t) + b_\delta \varepsilon_t \\
Y(t + \delta) &= c X(t + \delta) + d \epsilon_t,
\end{align}

where
\(\left\lbrace \varepsilon_t, \epsilon_t \right\rbrace_{t\in \mathbf{T}}\)
is a sequence of mutually independant standard Gaussian random
variables.

Following the papers notations, we have that:

\begin{align*}
q(x_t, x_{t + \delta}) &=\frac{1}{\sqrt{2\pi}b_\delta}\exp\left\lbrace-\frac{1}{2b^2_\delta} \left( x_{t+\delta} - a_\delta x_t \right)^2 \right\rbrace\\
g(x_{t + \delta}, y_{t + \delta}) &= \frac{1}{\sqrt{2\pi}d} \exp\left\lbrace -\frac{1}{2d^2} \left( y_{t+\delta} - c x_{t + \delta} \right)^2\right\rbrace
\end{align*}

For this model, we define the unnormalized kernel \(\mathbf{L}_n\)
\[\mathsf{X} \times \mathcal{X} \owns (x_{t_n}, A) \mapsto \int_A \ell(x_{t_n}, x_{t_n + \delta}, y_{t_n + \delta}) \text{d} x_{t + \delta}\]
where
\(\ell(x_t, x_{t + \delta}, y_{t + \delta}) = q(x_t, x_{t + \delta})g(x_{t + \delta}, y_{t + \delta})\).
We here note that \(\mathbf{L}_n\) is implicitly dependant on
\(y_{t_n + \delta}\) and on all the model parameters
\(\theta = \left(a_\delta, b_\delta, c, d \right)\).

By standard conjugation of Gaussian random variables, one can check that
we have:
\[\mathbf{L}_n\mathbf{1}_\mathsf{X}(x_{t_n}) = \frac{\gamma_\delta}{\sqrt{2 \pi}b_\delta d}\exp\left\lbrace -\frac{1}{2}\left(\frac{y^2_{t_n + \delta}}{d^2} + \frac{a_\delta^2x_{t_n}^2}{b_\delta^2}  - \gamma_\delta^2\times \left(\frac{c y_{t_n + \delta}}{d^2} + \frac{a_\delta x_{t_n}}{b_\delta^2} \right)^2 \right)   \right\rbrace, \]
where
\[\gamma^2_\delta = \left(\frac{1}{b_\delta^2 + \frac{c^2}{d^2}} \right)^{-1}\]

And we then have, for any function \(h \in \mathsf{F}(\mathcal{X})\):
\[\mathbf{L}_n h (x_{t_n}) = \mathbf{L}_n\mathbf{1}_\mathsf{X}(x_{t_n}) \times \mathbb{E}\left[h(X)\right],\]
where
\[X \sim \mathcal{N}\left(\gamma^2_\delta \left(\frac{c y_{t_n + \delta}}{d^2} + \frac{a_\delta x_{t_n}}{b_\delta^2} \right); \gamma_\delta^2\right).\]

\hypertarget{distance-between-original-and-skewed-model}{%
\section{Distance between original and skewed
model}\label{distance-between-original-and-skewed-model}}

Suppose that: \[
\text{d} X(t) = -\rho X(t) \text{d}t + \text{d}W(t),~X(0) = x(0),~t\geq 0.
\] Moreover, we suppose this process is observed at times
\(\mathbf{T} = (0, \delta, \dots, n\delta)\) with a centered error of
variance 1. This leads to the following model, that we'll denote
\(M_\star\)

\begin{align}
X(t + \delta) &=  \overbrace{\text{e}^{-\rho\delta}}^{a_\delta} X(t) + \overbrace{\sqrt{\frac{1}{2\rho}\left(1 - \text{e}^{-2\rho\delta}\right)}}^{b_\delta} \varepsilon_t \\
Y(t + \delta) &= \underbrace{1}_c \times X(t + \delta) + \underbrace{1}_d \times \epsilon_t,
\end{align}

Suppose that we would use an Euler scheme (with time step \(\delta\)) to
approximate the transition of the orginal model, suppose, moreover, that
the \(c\) parameter is misspecified and set to \(1 -\nu\)\^{}for some
\(0< \nu < 1\). We therefore have the following skewed model, denoted
\(M_{\delta, \nu}\): \begin{align}
X(t + \delta) &=  \overbrace{\left(1 - \rho\delta\right)}^{a_\delta} X(t) + \overbrace{\sqrt{\delta}}^{b_\delta} \varepsilon_t \\
Y(t + \delta) &= \underbrace{\left(1 - \nu\right)}_c \times X(t + \delta) + \underbrace{1}_d \times \epsilon_t,
\end{align}

\hypertarget{section}{%
\section{}\label{section}}

Consider the following model:

\begin{align}
X(t + \delta) &= \text{e}^{-\frac{1}{2} \delta} X(t) + \sqrt{1 - \text{e}^{-\delta}} E_t \\
Y(t + \delta) &= ( 1 - \varepsilon) \times X(t + \delta) + E^{'}_t,
\end{align}

where \(\left\lbrace E_t, E^{'}_t \right\rbrace_{t\in \mathbf{T}}\) is a
sequence of mutually independant standard Gaussian random variables, and
\(\varepsilon \in [0,1]\) is a bias parameter:

Following the paper notations, we have that:

\begin{align*}
q(x_t, x_{t + \delta}) &=\frac{1}{\sqrt{2\pi\times (1 - \text{e}^{-\delta})}}\exp\left\lbrace-\frac{1}{2(1 - \text{e}^{-\delta})} \left( x_{t+\delta} - \text{e}^{-\frac{1}{2} \delta} x_t \right)^2 \right\rbrace\\
g(x_{t + \delta}, y_{t + \delta}) &= \frac{1}{\sqrt{2\pi}} \exp\left\lbrace -\frac{1}{2} \left( y_{t+\delta} - (1 - \varepsilon) x_{t + \delta} \right)^2\right\rbrace
\end{align*}

\end{document}
