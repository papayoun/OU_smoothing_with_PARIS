\documentclass[]{article}
\usepackage{lmodern}
\usepackage{amssymb,amsmath}
\usepackage{ifxetex,ifluatex}
\usepackage{fixltx2e} % provides \textsubscript
\ifnum 0\ifxetex 1\fi\ifluatex 1\fi=0 % if pdftex
  \usepackage[T1]{fontenc}
  \usepackage[utf8]{inputenc}
\else % if luatex or xelatex
  \ifxetex
    \usepackage{mathspec}
  \else
    \usepackage{fontspec}
  \fi
  \defaultfontfeatures{Ligatures=TeX,Scale=MatchLowercase}
\fi
% use upquote if available, for straight quotes in verbatim environments
\IfFileExists{upquote.sty}{\usepackage{upquote}}{}
% use microtype if available
\IfFileExists{microtype.sty}{%
\usepackage[]{microtype}
\UseMicrotypeSet[protrusion]{basicmath} % disable protrusion for tt fonts
}{}
\PassOptionsToPackage{hyphens}{url} % url is loaded by hyperref
\usepackage[unicode=true]{hyperref}
\hypersetup{
            pdftitle={Smoothing in partially observed OU process},
            pdfauthor={Pierre Gloaguen},
            pdfborder={0 0 0},
            breaklinks=true}
\urlstyle{same}  % don't use monospace font for urls
\usepackage[margin=1in]{geometry}
\usepackage{graphicx,grffile}
\makeatletter
\def\maxwidth{\ifdim\Gin@nat@width>\linewidth\linewidth\else\Gin@nat@width\fi}
\def\maxheight{\ifdim\Gin@nat@height>\textheight\textheight\else\Gin@nat@height\fi}
\makeatother
% Scale images if necessary, so that they will not overflow the page
% margins by default, and it is still possible to overwrite the defaults
% using explicit options in \includegraphics[width, height, ...]{}
\setkeys{Gin}{width=\maxwidth,height=\maxheight,keepaspectratio}
\IfFileExists{parskip.sty}{%
\usepackage{parskip}
}{% else
\setlength{\parindent}{0pt}
\setlength{\parskip}{6pt plus 2pt minus 1pt}
}
\setlength{\emergencystretch}{3em}  % prevent overfull lines
\providecommand{\tightlist}{%
  \setlength{\itemsep}{0pt}\setlength{\parskip}{0pt}}
\setcounter{secnumdepth}{0}
% Redefines (sub)paragraphs to behave more like sections
\ifx\paragraph\undefined\else
\let\oldparagraph\paragraph
\renewcommand{\paragraph}[1]{\oldparagraph{#1}\mbox{}}
\fi
\ifx\subparagraph\undefined\else
\let\oldsubparagraph\subparagraph
\renewcommand{\subparagraph}[1]{\oldsubparagraph{#1}\mbox{}}
\fi

% set default figure placement to htbp
\makeatletter
\def\fps@figure{htbp}
\makeatother


\title{Smoothing in partially observed OU process}
\author{Pierre Gloaguen}
\date{25/11/2020}

\begin{document}
\maketitle

\section{True model}\label{true-model}

\begin{align}
\text{d} X(t) &= -\rho(X(t) - \mu) \text{d}t + \sigma \text{d}W(t),~X(0) = x(0)~t\geq 0\\
Y(k) &= X(k) + \sigma_{obs}\varepsilon_k,~ \varepsilon_k \overset{i.i.d}{\sim} \mathcal{N}(0, 1)
\end{align}

\begin{itemize}
\tightlist
\item
  Data are generated according to this true model, giving a vector
  \(y_{0:T}\) from which inference must be performed.
\end{itemize}

\section{Biased model}\label{biased-model}

We consider a bias version of the true model. This model is biased in
two ways:

\begin{enumerate}
\def\labelenumi{\arabic{enumi}.}
\tightlist
\item
  In the dynamics model, which is approximated through an Euler scheme
  (and setting that \(t_{k + 1} = t_k + \delta\)).
\item
  In the observation model, where it is supposed that only a fraction
  \(\alpha_{obs}\) of \(X(t_k)\) is observed.

  \begin{align}
  X(t_k + \delta) &= -\rho(X(t_k) - \mu)\times \delta + \sigma\sqrt{\delta} \times \tilde{\varepsilon}_k, X(0) = x(0), ~k =  0, 1,..., \tilde{\varepsilon}_k \overset{i.i.d}{\sim} \mathcal{N}(0, 1)\\
  Y(t_k) &= \alpha_{obs}X(t_k) + \sigma_{obs}\varepsilon_k,~k = 0, 1, ..., \varepsilon_k \overset{i.i.d}{\sim} \mathcal{N}(0, 1)
  \end{align}
\end{enumerate}

We here emphasize that this scenario is actually quite common, for
instance in ecology, where the continuous time dynamics model is
approximated in discrete time, and supposed observation model is subject
to some bias (we refer the interested reader to the literature of state
space models for population dynamics).

\section{Smoothing}\label{smoothing}

\begin{itemize}
\item
  In this context, our goal is to estimate, for each \(n\leq T\) the
  realization on \(y_{0:n}\) of the random variable:
  \[\Phi_n = \mathbb{E}\left[\sum_{k = 0}^{n} X(t_k)\vert Y_{0:n}\right].\]
\item
  It is known that in this context, the actual marginal distribution is
  Gaussian with parameters that can be obtained through Kalman
  recursions. The quantity \(\Phi_n\) can then be computed explicitly.
\item
  We consider three alternatives estimation of \(\Phi_n\) which are:

  \begin{itemize}
  \tightlist
  \item
    \emph{PaRIS:} The online estimation is obtained with the PaRIS
    method. The model used in the recursions is the true one, and the
    proposal is the optimal filter;
  \item
    \emph{Biased PaRIS:} The online estimation is obtained with the
    PaRIS method. However, here, the model used in the recursions is the
    biased one, and the proposal is the optimal filter relative to this
    biased model;
  \item
    \emph{Biased Kalman:} The estimation obtained with the Kalman
    recursions when the true model is replaced by the biased model;
  \end{itemize}
\item
  For each estimator \(\hat{\Phi}_n\), we compute
\item
  The scenario is performed with
  \(T = 100,~\delta = 0.5,~\rho = 1,~\mu = -5,~x(0) = 10,~\sigma = 0.5,~\sigma_{obs} = 0.7,~\alpha_{obs} = 0.95\),
  using 200 particles for the particle filter, and obtaining two
  backward samples at each backward sampling step.
\end{itemize}

\section{Results}\label{results}

\begin{center}\includegraphics{experiment_settings_files/figure-latex/plot_results-1} \end{center}

We can see here that using the biased model (either with Kalman
recursions or PaRIS), the bias is our estimate only grows linearly with
\(n\), as expected by results EQREF. In this context, the bias due to
Monte Carlo error is not visible.

\section{Bias in the observation
model}\label{bias-in-the-observation-model}

Consider the following model:

\begin{align}
X(t + \delta) &= \text{e}^{-\frac{1}{2} \delta} X(t) + \sqrt{1 - \text{e}^{-\delta}} E_t \\
Y(t + \delta) &= ( 1 - \varepsilon) \times X(t + \delta) + E^{'}_t,
\end{align}

where \(\left\lbrace E_t, E^{'}_t \right\rbrace_{t\in \mathbf{T}}\) is a
sequence of mutually independant standard Gaussian random variables, and
\(\varepsilon \in [0,1]\) is a bias parameter:

Following the paper notations, we have that:

\begin{align*}
q(x_t, x_{t + \delta}) &=\frac{1}{\sqrt{2\pi\times (1 - \text{e}^{-\delta})}}\exp\left\lbrace-\frac{1}{2(1 - \text{e}^{-\delta})} \left( x_{t+\delta} - \text{e}^{-\frac{1}{2} \delta} x_t \right)^2 \right\rbrace\\
g(x_{t + \delta}, y_{t + \delta},\varepsilon) &= \frac{1}{\sqrt{2\pi}} \exp\left\lbrace -\frac{1}{2} \left( y_{t+\delta} - (1 - \varepsilon) x_{t + \delta} \right)^2\right\rbrace;
\end{align*}

One can see that \(g(x_{t + \delta}, y_{t + \delta}, \varepsilon)\) is
Lipschitz in its \(\varepsilon\) (as its derivative is bounded)
argument, and morever that :
\[\left\vert g(x_{t + \delta}, y_{t + \delta},\varepsilon) - g(x_{t + \delta}, y_{t + \delta},0)\right\vert \leq \sup_{0\leq \varepsilon \leq 1} \left\vert\frac{\delta}{\delta \varepsilon} g(x_{t + \delta}, y_{t + \delta},\varepsilon)\right\vert \times \varepsilon
\leq \overbrace{\left( \left\vert x_{t + \delta}y_{t + \delta} - x^2_{t + \delta} \right\vert + x^2_{t + \delta}\right)}^{:= k(x_{t + \delta}, y_{t + \delta})} \varepsilon.\]
Therefore, we have that:

\begin{align*}
\left\vert \mathbf{L}_t^\varepsilon h(x_t) - \mathbf{L}_t h(x_t)  \right\vert &= \left\vert \int q(x_t, x_{t + \delta} )\left( g(x_{t + \delta}, y_{t + \delta},\varepsilon) - g(x_{t + \delta}, y_{t + \delta}, 0)\right) h(x_{t+\delta})\text{d}x_{t + \delta}  \right\vert\\
&\leq \varepsilon \int q(x_t, x_{t + \delta} ) k(x_{t + \delta},y_{t + \delta}) \vert h(x_{t+\delta})\vert  \text{d} x_{t + \delta}.
\end{align*}

Therefore, the integral must be uniformly bounded for every
\(x_\delta\). In the case where
\(h(x) = x \mathbf{1}_{\vert x \vert \leq M}\), then
\(k(x, y)\vert h(x) \vert \leq M(2M^2 + M\vert y\vert)\), and we have
\[\left\vert \mathbf{L}_t^\varepsilon h(x_t) - \mathbf{L}_t h(x_t)  \right\vert \leq M(2M^2 + M\vert y_{t + \delta}\vert)\varepsilon\]

\end{document}
